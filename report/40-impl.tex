\chapter{Технологическая часть}
\label{cha:impl}

В данной части будет описан подход для программной реализации ранее указанных алгоритмов.

\section{Требования к программному обеспечению}

Разработанное ПО должно предоставлять возможность замеров процессорного времени
выполнения каждого алгоритма. Требуется вводить 2 строки и выводить матрицу (кроме рекурсивной реализации) и значения расстояний, полученных различными реализациями.

\section{Средства реализации}

Для реализации ПО был выбран язык Go, поскольку в его основном пакете присутствуют
удобные средства замера процессорного времени и памяти модуля testing.

\section{Листинг кода}

В данном разделе приведены листинги реализаций ранее указанных алгоритмов на языке Go.

\lstset{language=Golang}
\begin{lstlisting}[caption={Матричная версия алгоритма Дамерау-Левенштейна}]
func LevenshteinDamerau(first string, second string) (int, [][] int) {
	lenFirst := len(first)
	lenSecond := len(second)
	rows := lenFirst + 1
	cols := lenSecond + 1
	matrix := allocateMatrix(rows, cols)

	for i := 0; i < rows; i++ {
		matrix[i][0] = i
	}
	for i := 1; i < cols; i++ {
		matrix[0][i] = i
	}

	add := 0
	for j := 1; j < cols; j++ {
		for i := 1; i < rows; i++ {
			if first[i-1] == second[j-1] {
				add = 0
			} else {
				add = 1
			}
			matrix[i][j] = MinThree(matrix[i-1][j]+1,
				matrix[i][j-1]+1,
				matrix[i-1][j-1]+add)

			if i > 1 && j > 1 && first[i-1] == second[j-2] && first[i-2] == second[j-1] {
				matrix[i][j] = Min(matrix[i][j], matrix[i-2][j-2]+add)
			}
		}
	}

	return matrix[lenFirst][lenSecond], matrix
}
\end{lstlisting}

\begin{lstlisting}[caption={Матричная реализация алгоритма Левенштейна}]
func LevenshteinIterative(first string, second string) (int, [][]int) {
	lenFirst := len(first)
	lenSecond := len(second)
	rows := lenFirst + 1
	cols := lenSecond + 1
	matrix := allocateMatrix(rows, cols)

	for i := 0; i < rows; i++ {
		matrix[i][0] = i
	}
	for i := 1; i < cols; i++ {
		matrix[0][i] = i
	}

	add := 0
	for j := 1; j < cols; j++ {
		for i := 1; i < rows; i++ {
			if first[i-1] == second[j-1] {
				add = 0
			} else {
				add = 1
			}
			matrix[i][j] = MinThree(matrix[i-1][j]+1,
				matrix[i][j-1]+1,
				matrix[i-1][j-1]+add)
		}
	}

	return matrix[lenFirst][lenSecond], matrix
}
\end{lstlisting}

\begin{lstlisting}[caption={Рекурсивная версия алгоритма Левенштейна}]
func LevenshteinRecursive(first string, second string) int {
    lenFirst := len(first)
    lenSecond := len(second)
    if Min(lenFirst, lenSecond) == 0 {
        return Max(lenFirst, lenSecond)
    } else {
        add := 1
        if first[lenFirst-1] == second[lenSecond-1] {
            add = 0
        }

        return MinThree(
            LevenshteinRecursive(first[:lenFirst-1], second)+1,
            LevenshteinRecursive(first, second[:lenSecond-1])+1,
            LevenshteinRecursive(first[:lenFirst-1], second[:lenSecond-1])+add)
    }
}
\end{lstlisting}

\begin{lstlisting}[caption={Оптимизированная версия рекурсивного алгоритма Левенштейна}]
func levenshteinRecursiveModule(first, second string, result, minval int) int {
	lenFirst := len(first)
	lenSecond := len(second)
	if result >= minval {
		return minval
	} else if lenFirst == 0 {
		return result +lenSecond
	} else if lenSecond == 0 {
		return result + lenFirst
	} else {
		add := 1
		if first[lenFirst-1] == second[lenSecond-1] {
			add = 0
		}
		r1 := levenshteinRecursiveModule(first[:lenFirst-1], second[:lenSecond-1], result+add, minval)
		r2 := levenshteinRecursiveModule(first, second[:lenSecond-1], result+1, Min(r1, minval))
		r3 := levenshteinRecursiveModule(first[:lenFirst-1], second, result+1, MinThree(r1, r2, minval))
		return MinThree(r1, r2, r3)
	}
}

func LevenshteinRecursiveOptimized(first string, second string) int {
	minval := Max(len(first), len(second))
	result := 0
	return levenshteinRecursiveModule(first, second, result, minval)
}
\end{lstlisting}

\section{Описание тестирования}
% описать, какие тесты будут проведены

Тестирование будет проведено для каждой из реализаций со следующими входными данными:
\begin{itemize}
	\item пустые строки;
	\item идентичные строки;
	\item строки одинаковой длины, но с разными символами;
	\item строки различной длины, но отличных лишь в некотором числе вставок или удалений символа;
	\item строки различной длины и требующих замены символа.
\end{itemize}

Для расстояния Дамерау-Левенштейна дополнительный тест будет заключаться в перестановке двух соседних символов.
%%%% mode: latex
%%%% TeX-master: "rpz"
%%%% End:
