
%\NirEkz{Экз. 3}                                  % Раскоментировать если не требуется
%\NirGrif{Секретно}                % Наименование грифа

\gosttitle{GostRV15-110}       % Шаблон титульной страницы, по умолчанию будет ГОСТ 7.32-2001,
% Варианты GostRV15-110 или Gost7-32

\NirOrgLongName{
МОСКОВСКИЙ ГОСУДАРСТВЕННЫЙ ТЕХНИЧЕСКИЙ УНИВЕРСИТЕТ ИМ. Н. Э. БАУМАНА
}                                           %% Полное название организации

% \NirUdk{УДК № 004.822}
% \NirGosNo{№ госрегистрации }
% \NirInventarNo{Инв. № ??????}

%\NirConfirm{Согласовано}                  % Смена УТВЕРЖДАЮ
%\NirBoss[.49]{Проректор университета\\по научной работе}{В.Н. Зимин.}            %% Заказчик, утверждающий НИР


\NirReportName{Отчёт по дисциплине "Анализ алгоритмов"}   % Можно поменять тип отчета
\NirAbout{О лабораторной работе \No{1}} %Можно изменить о чем отчет

%\NirPartNum{Часть}{1}                      % Часть номер

%\NirBareSubject{}                  % Убирает по теме если раскоментить

% \NirIsAnnotacion{АННОТАЦИОННЫЙ }         %% Раскомментируйте, если это аннотационный отчёт
%\NirStage{промежуточный}{Этап \No 1}{} %%% Этап НИР: {номер этапа}{вид отчёта - промежуточный или заключительный}{название этапа}
%\NirStage{}{}{} %%% Этап НИР: {номер этапа}{вид отчёта - промежуточный или

% \Nir{Анализ алгоритмов}

\NirSubject{ "Расстояние Левенштейна и Дамерау-Левенштейна"}                                   % Наименование темы
%\NirFinal{}                        % Заключительный, если закоментировать то промежуточный
%\finalname{итоговый}               % Название финального отчета (Заключительный)
%\NirCode{Шифр\,---\,САПР-РЛС-ФИЗТЕХ-1} % Можно задать шифр как в ГОСТ 15.110
\NirCode{}

% \NirManager{H}{Р.А. Бадамшин  } %% Название руководителя
\NirIsp{Студент ИУ7-53Б}{Пудов Дмитрий Юрьевич} %% Название руководителя

% \NirYear{1999}%% если нужно поменять год отчёта; если закомментировано, ставится текущий год
\NirTown{Москва}                           %% город, в котором написан отчёт
