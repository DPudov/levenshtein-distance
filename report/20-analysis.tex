\chapter{Аналитическая часть}
\label{cha:analysis}

В данной части будут описаны суть задач нахождения расстояния Левенштейна и Дамерау-Левенштейна, а также математические способы их решения.

Поиск расстояния Левенштейна заключается в определении минимального количества редакционных операций (вставка, удаление, замена одного символа), необходимых для трансформации одной строки в другую.

В задаче о расстоянии Дамерау-Левенштейна во множество редакционных операций добавляется транспозиция - перестановка двух соседних символов.

\section{Описание алгоритмов}

Пусть $a$ и $b$ - строки над некоторым алфавитом длины M и N соответственно. Тогда расстояние Левенштейна определяется формулой $d(a, b) = D(M, N)$, где
\begin{equation*}
    D(i, j) =
    \begin{cases}
        0 & i=0, j=0\\
        i & j=0, i>0\\
        j & i=0, j>0\\
        min
        \begin{cases}
            D(i, j-1) + 1\\
            D(i-1, j) + 1\\
            D(i-1, j-1) + 1_{a[i] \ne b[j]}\\
        \end{cases}
    \end{cases}
\end{equation*}

Расстояние Дамерау-Левенштейна для данных строк определяется как $d(a, b) = D(M, N)$, где
\begin{equation*}
    D(i, j) =
    \begin{cases}
        max(i, j) & min(i, j)=0\\
        min
        \begin{cases}
            D(i, j)+1\\
            D(i, j-1)+1\\
            D(i-1, j-1)+1_{a[i] \ne b[j]}\\
            D(i-2, j-2)+1\\
        \end{cases} & i>1 \text{ и } j>1 \\ \text{ и } a[i]=b[j-1] \text{ и } a[i-1]=b[j]\\
        min
        \begin{cases}
            D(i-1, j)+1\\
            D(i, j-1)+1\\
            D(i-1, j-1)+1_{a[i] \ne b[j]}\\
        \end{cases} & \text{иначе.}
    \end{cases}
\end{equation*}
