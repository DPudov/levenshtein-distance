%% Преамбула TeX-файла

% 1. Стиль и язык
\documentclass[utf8x, 14pt]{G7-32} % Стиль (по умолчанию будет 14pt)

% Остальные стандартные настройки убраны в preamble.inc.tex.
\sloppy

% Настройки стиля ГОСТ 7-32
% Для начала определяем, хотим мы или нет, чтобы рисунки и таблицы нумеровались в пределах раздела, или нам нужна сквозная нумерация.
\EqInChapter % формулы будут нумероваться в пределах раздела
\TableInChapter % таблицы будут нумероваться в пределах раздела
\PicInChapter % рисунки будут нумероваться в пределах раздела

% Добавляем гипертекстовое оглавление в PDF
\usepackage[
bookmarks=true, colorlinks=true, unicode=true,
urlcolor=black,linkcolor=black, anchorcolor=black,
citecolor=black, menucolor=black, filecolor=black,
]{hyperref}
\usepackage{pgfplots}
\usepackage{nomencl}
\usepackage{placeins}
\usepackage{float}

\AfterHyperrefFix

\usepackage{microtype}% полезный пакет для микротипографии, увы под xelatex мало чего умеет, но под pdflatex хорошо улучшает читаемость

% Тире могут быть невидимы в Adobe Reader
\ifInvisibleDashes
\MakeDashesBold
\fi

\usepackage{graphicx}   % Пакет для включения рисунков

% С такими оно полями оно работает по-умолчанию:
% \RequirePackage[left=20mm,right=10mm,top=20mm,bottom=20mm,headsep=0pt,includefoot]{geometry}
% Если вас тошнит от поля в 10мм --- увеличивайте до 20-ти, ну и про переплёт не забывайте:
\geometry{right=10mm}
\geometry{left=30mm}
\geometry{bottom=20mm}
\geometry{ignorefoot}% считать от нижней границы текста


% Пакет Tikz
\usepackage{tikz}
\usetikzlibrary{arrows,positioning,shadows}

\usepackage{epstopdf}
% Произвольная нумерация списков.
\usepackage{enumerate}

% ячейки в несколько строчек
\usepackage{multirow}

% itemize внутри tabular
\usepackage{paralist,array}

%\setlength{\parskip}{1ex plus0.5ex minus0.5ex} % разрыв между абзацами
\setlength{\parskip}{1ex} % разрыв между абзацами
\usepackage{blindtext}

% Центрирование подписей к плавающим окружениям
%\usepackage[justification=centering]{caption}

\usepackage{newfloat}
\DeclareFloatingEnvironment[
placement={!ht},
name=Equation
]{eqndescNoIndent}
\edef\fixEqndesc{\noexpand\setlength{\noexpand\parindent}{\the\parindent}\noexpand\setlength{\noexpand\parskip}{\the\parskip}}
\newenvironment{eqndesc}[1][!ht]{%
    \begin{eqndescNoIndent}[#1]%
\fixEqndesc%
}
{\end{eqndescNoIndent}}

\usepackage{afterpage}

\newcommand\blankpage{
	\null
	\thispagestyle{empty}
	\newpage
}

\pgfplotsset{compat=1.15}


% Настройки листингов.
\ifPDFTeX
% 8 Листинги

\usepackage{listings}

% Значения по умолчанию
\lstset{
  basicstyle= \footnotesize,
  breakatwhitespace=true,% разрыв строк только на whitespacce
  breaklines=true,       % переносить длинные строки
%   captionpos=b,          % подписи снизу -- вроде не надо
  inputencoding=koi8-r,
  numbers=left,          % нумерация слева
  numberstyle=\footnotesize,
  showspaces=false,      % показывать пробелы подчеркиваниями -- идиотизм 70-х годов
  showstringspaces=false,
  showtabs=false,        % и табы тоже
  stepnumber=1,
  tabsize=4,              % кому нужны табы по 8 символов?
  frame=single,
  xleftmargin=2.4em,
  framexleftmargin=2em
}

% Стиль для псевдокода: строчки обычно короткие, поэтому размер шрифта побольше
\lstdefinestyle{pseudocode}{
  basicstyle=\small,
  keywordstyle=\color{black}\bfseries\underbar,
  language=Pseudocode,
  numberstyle=\footnotesize,
  commentstyle=\footnotesize\it
}

% Стиль для обычного кода: маленький шрифт
\lstdefinestyle{realcode}{
  basicstyle=\scriptsize,
  numberstyle=\footnotesize
}

% Стиль для коротких кусков обычного кода: средний шрифт
\lstdefinestyle{simplecode}{
  basicstyle=\footnotesize,
  numberstyle=\footnotesize
}

% Стиль для BNF
\lstdefinestyle{grammar}{
  basicstyle=\footnotesize,
  numberstyle=\footnotesize,
  stringstyle=\bfseries\ttfamily,
  language=BNF
}

% Определим свой язык для написания псевдокодов на основе Python
\lstdefinelanguage[]{Pseudocode}[]{Python}{
  morekeywords={each,empty,wait,do},% ключевые слова добавлять сюда
  morecomment=[s]{\{}{\}},% комменты {а-ля Pascal} смотрятся нагляднее
  literate=% а сюда добавлять операторы, которые хотите отображать как мат. символы
    {->}{\ensuremath{$\rightarrow$}~}2%
    {<-}{\ensuremath{$\leftarrow$}~}2%
    {:=}{\ensuremath{$\leftarrow$}~}2%
    {<--}{\ensuremath{$\Longleftarrow$}~}2%
}[keywords,comments]

% Свой язык для задания грамматик в BNF
\lstdefinelanguage[]{BNF}[]{}{
  morekeywords={},
  morecomment=[s]{@}{@},
  morestring=[b]",%
  literate=%
    {->}{\ensuremath{$\rightarrow$}~}2%
    {*}{\ensuremath{$^*$}~}2%
    {+}{\ensuremath{$^+$}~}2%
    {|}{\ensuremath{$|$}~}2%
}[keywords,comments,strings]

% Подписи к листингам на русском языке.
\renewcommand\lstlistingname{Листинг}
\renewcommand\lstlistlistingname{Листинги}

\else
\usepackage{local-minted}
\fi

% Полезные макросы листингов.
% Любимые команды
\newcommand{\Code}[1]{\textbf{#1}}


% Стиль титульного листа и заголовки

%\NirEkz{Экз. 3}                                  % Раскоментировать если не требуется
%\NirGrif{Секретно}                % Наименование грифа

\gosttitle{GostRV15-110}       % Шаблон титульной страницы, по умолчанию будет ГОСТ 7.32-2001,
% Варианты GostRV15-110 или Gost7-32

\NirOrgLongName{
МОСКОВСКИЙ ГОСУДАРСТВЕННЫЙ ТЕХНИЧЕСКИЙ УНИВЕРСИТЕТ ИМ. Н. Э. БАУМАНА
}                                           %% Полное название организации

% \NirUdk{УДК № 004.822}
% \NirGosNo{№ госрегистрации }
% \NirInventarNo{Инв. № ??????}

%\NirConfirm{Согласовано}                  % Смена УТВЕРЖДАЮ
%\NirBoss[.49]{Проректор университета\\по научной работе}{В.Н. Зимин.}            %% Заказчик, утверждающий НИР


\NirReportName{Отчёт по дисциплине "Анализ алгоритмов"}   % Можно поменять тип отчета
\NirAbout{О лабораторной работе \No{1}} %Можно изменить о чем отчет

%\NirPartNum{Часть}{1}                      % Часть номер

%\NirBareSubject{}                  % Убирает по теме если раскоментить

% \NirIsAnnotacion{АННОТАЦИОННЫЙ }         %% Раскомментируйте, если это аннотационный отчёт
%\NirStage{промежуточный}{Этап \No 1}{} %%% Этап НИР: {номер этапа}{вид отчёта - промежуточный или заключительный}{название этапа}
%\NirStage{}{}{} %%% Этап НИР: {номер этапа}{вид отчёта - промежуточный или

% \Nir{Анализ алгоритмов}

\NirSubject{ "Расстояние Левенштейна и Дамерау-Левенштейна"}                                   % Наименование темы
%\NirFinal{}                        % Заключительный, если закоментировать то промежуточный
%\finalname{итоговый}               % Название финального отчета (Заключительный)
%\NirCode{Шифр\,---\,САПР-РЛС-ФИЗТЕХ-1} % Можно задать шифр как в ГОСТ 15.110
\NirCode{}

% \NirManager{H}{Р.А. Бадамшин  } %% Название руководителя
\NirIsp{Студент ИУ7-53Б}{Пудов Дмитрий Юрьевич} %% Название руководителя

% \NirYear{1999}%% если нужно поменять год отчёта; если закомментировано, ставится текущий год
\NirTown{Москва}                           %% город, в котором написан отчёт



\begin{document}

\frontmatter % выключает нумерацию ВСЕГО; здесь начинаются ненумерованные главы: реферат, введение, глоссарий, сокращения и прочее.

\maketitle %создает титульную страницу
% \afterpage{\blankpage}\afterpage{\blankpage}\afterpage{\blankpage}
% пропущены страницы под тз и план (у меня их 2 тз и 1 план, итого 3, вам надо пропустить столько, сколько страниц у вас в тз и плане)


%\listoffigures                         % Список рисунков

%\listoftables                          % Список таблиц

%\NormRefs % Нормативные ссылки
% Команды \breakingbeforechapters и \nonbreakingbeforechapters
% управляют разрывом страницы перед главами.
% По-умолчанию страница разрывается.

% \Referat
%\begin{abstract}

    Отчет содержит \pageref{LastPage}\,стр.%
    \ifnum \totfig >0
    , \totfig~рис.%
    \fi
    \ifnum \tottab >0
    , \tottab~табл.%
    \fi
    %
    \ifnum \totbib >0
    , \totbib~источн.%
    \fi
    %
    \ifnum \totapp >0
    , \totapp~прил.%
    \else
    .%
    \fi


%\end{abstract}

%%% Local Variables: 
%%% mode: latex
%%% TeX-master: "rpz"
%%% End: 

% \nobreakingbeforechapters
% \breakingbeforechapters

\tableofcontents

% \printnomenclature % Автоматический список сокращений

\Introduction

Цель работы: изучение метода динамического программирования на материале
алгоритмов  Левенштейна и Дамерау-Левенштейна.

Постановка задачи:

\begin{itemize}
\item изучить метод метод динамического программирования на материала алгоритмов Левенштейна и Дамерау-Левенштейна;
\item применить его;
\item получить практические навыки реализации указанных алгоритмов.
\end{itemize}


\mainmatter % это включает нумерацию глав и секций в документе ниже

\chapter{Аналитическая часть}
\label{cha:analysis}

В данной части будут описаны суть задач нахождения расстояния Левенштейна и Дамерау-Левенштейна, а также математические способы их решения.

Поиск расстояния Левенштейна заключается в определении минимального количества редакционных операций (вставка, удаление, замена одного символа), необходимых для трансформации одной строки в другую.

В задаче о расстоянии Дамерау-Левенштейна во множество редакционных операций добавляется транспозиция - перестановка двух соседних символов.

\section{Описание алгоритмов}

Пусть $a$ и $b$ - строки над некоторым алфавитом длины M и N соответственно. Тогда расстояние Левенштейна определяется формулой $d(a, b) = D(M, N)$, где
\begin{equation*}
    D(i, j) =
    \begin{cases}
        0 & i=0, j=0\\
        i & j=0, i>0\\
        j & i=0, j>0\\
        min
        \begin{cases}
            D(i, j-1) + 1\\
            D(i-1, j) + 1\\
            D(i-1, j-1) + 1_{a[i] \ne b[j]}\\
        \end{cases}
    \end{cases}
\end{equation*}

Расстояние Дамерау-Левенштейна для данных строк определяется как $d(a, b) = D(M, N)$, где
\begin{equation*}
    D(i, j) =
    \begin{cases}
        max(i, j) & min(i, j)=0\\
        min
        \begin{cases}
            D(i, j)+1\\
            D(i, j-1)+1\\
            D(i-1, j-1)+1_{a[i] \ne b[j]}\\
            D(i-2, j-2)+1\\
        \end{cases} & i>1 \text{ и } j>1 \\ \text{ и } a[i]=b[j-1] \text{ и } a[i-1]=b[j]\\
        min
        \begin{cases}
            D(i-1, j)+1\\
            D(i, j-1)+1\\
            D(i-1, j-1)+1_{a[i] \ne b[j]}\\
        \end{cases} & \text{иначе.}
    \end{cases}
\end{equation*}

\chapter{Конструкторская часть}
\label{cha:design}

\section{Разработка алгоритмов}
% сюда схемы алгоритмов

\section{Сравнительный анализ рекурсивной и нерекурсивной\\ реализаций}

Какой-то текст

\chapter{Технологическая часть}
\label{cha:impl}

В данной части будет описан подход для программной реализации ранее указанных алгоритмов.

\section{Требования к программному обеспечению}

Разработанное ПО должно предоставлять возможность замеров процессорного времени
выполнения каждого алгоритма. Требуется вводить 2 строки и выводить матрицу (кроме рекурсивной реализации) и значения расстояний, полученных различными реализациями.

\section{Средства реализации}

Для реализации ПО был выбран язык Go, поскольку в его основном пакете присутствуют
удобные средства замера процессорного времени и памяти модуля testing.

\section{Листинг кода}

В данном разделе приведены листинги реализаций ранее указанных алгоритмов на языке Go.

\lstset{language=Golang}
\begin{lstlisting}[caption={Матричная версия алгоритма Дамерау-Левенштейна}]
func LevenshteinDamerau(first string, second string) (int, [][] int) {
	lenFirst := len(first)
	lenSecond := len(second)
	rows := lenFirst + 1
	cols := lenSecond + 1
	matrix := allocateMatrix(rows, cols)

	for i := 0; i < rows; i++ {
		matrix[i][0] = i
	}
	for i := 1; i < cols; i++ {
		matrix[0][i] = i
	}

	add := 0
	for j := 1; j < cols; j++ {
		for i := 1; i < rows; i++ {
			if first[i-1] == second[j-1] {
				add = 0
			} else {
				add = 1
			}
			matrix[i][j] = MinThree(matrix[i-1][j]+1,
				matrix[i][j-1]+1,
				matrix[i-1][j-1]+add)

			if i > 1 && j > 1 && first[i-1] == second[j-2] && first[i-2] == second[j-1] {
				matrix[i][j] = Min(matrix[i][j], matrix[i-2][j-2]+add)
			}
		}
	}

	return matrix[lenFirst][lenSecond], matrix
}
\end{lstlisting}

\begin{lstlisting}[caption={Матричная реализация алгоритма Левенштейна}]
func LevenshteinIterative(first string, second string) (int, [][]int) {
	lenFirst := len(first)
	lenSecond := len(second)
	rows := lenFirst + 1
	cols := lenSecond + 1
	matrix := allocateMatrix(rows, cols)

	for i := 0; i < rows; i++ {
		matrix[i][0] = i
	}
	for i := 1; i < cols; i++ {
		matrix[0][i] = i
	}

	add := 0
	for j := 1; j < cols; j++ {
		for i := 1; i < rows; i++ {
			if first[i-1] == second[j-1] {
				add = 0
			} else {
				add = 1
			}
			matrix[i][j] = MinThree(matrix[i-1][j]+1,
				matrix[i][j-1]+1,
				matrix[i-1][j-1]+add)
		}
	}

	return matrix[lenFirst][lenSecond], matrix
}
\end{lstlisting}

\begin{lstlisting}[caption={Рекурсивная версия алгоритма Левенштейна}]
func LevenshteinRecursive(first string, second string) int {
    lenFirst := len(first)
    lenSecond := len(second)
    if Min(lenFirst, lenSecond) == 0 {
        return Max(lenFirst, lenSecond)
    } else {
        add := 1
        if first[lenFirst-1] == second[lenSecond-1] {
            add = 0
        }

        return MinThree(
            LevenshteinRecursive(first[:lenFirst-1], second)+1,
            LevenshteinRecursive(first, second[:lenSecond-1])+1,
            LevenshteinRecursive(first[:lenFirst-1], second[:lenSecond-1])+add)
    }
}
\end{lstlisting}

\begin{lstlisting}[caption={Оптимизированная версия рекурсивного алгоритма Левенштейна}]
func levenshteinRecursiveModule(first, second string, result, minval int) int {
	lenFirst := len(first)
	lenSecond := len(second)
	if result >= minval {
		return minval
	} else if lenFirst == 0 {
		return result +lenSecond
	} else if lenSecond == 0 {
		return result + lenFirst
	} else {
		add := 1
		if first[lenFirst-1] == second[lenSecond-1] {
			add = 0
		}
		r1 := levenshteinRecursiveModule(first[:lenFirst-1], second[:lenSecond-1], result+add, minval)
		r2 := levenshteinRecursiveModule(first, second[:lenSecond-1], result+1, Min(r1, minval))
		r3 := levenshteinRecursiveModule(first[:lenFirst-1], second, result+1, MinThree(r1, r2, minval))
		return MinThree(r1, r2, r3)
	}
}

func LevenshteinRecursiveOptimized(first string, second string) int {
	minval := Max(len(first), len(second))
	result := 0
	return levenshteinRecursiveModule(first, second, result, minval)
}
\end{lstlisting}

\section{Описание тестирования}
% описать, какие тесты будут проведены

Тестирование будет проведено для каждой из реализаций со следующими входными данными:
\begin{itemize}
	\item пустых строки;
	\item идентичные строки;
	\item строки одинаковой длины, но с разными символами;
	\item строки различной длины, но отличных лишь в некотором числе вставок или удалений символа;
	\item строки различной длины и требующих замены символа.
\end{itemize}
%%%% mode: latex
%%%% TeX-master: "rpz"
%%%% End:

\chapter{Исследовательский раздел}
\label{cha:research}


\backmatter %% Здесь заканчивается нумерованная часть документа и начинаются ссылки и

\Conclusion % заключение к отчёту

%% заключение


% % Список литературы при помощи BibTeX
% Юзать так:
%
% pdflatex rpz
% bibtex rpz
% pdflatex rpz

\bibliographystyle{ugost2008}
\bibliography{rpz}

%%% Local Variables: 
%%% mode: latex
%%% TeX-master: "rpz"
%%% End: 


%
\appendix   % Тут идут приложения
%
\chapter{Реализация и тестирование}

%
%\chapter{Еще картинки}
\label{cha:appendix2}
\blindtext

\begin{figure}
\centering
\caption{Еще одна картинка, ничем не лучше предыдущей. Но надо же как-то заполнить место.}
\end{figure}

%%% Local Variables: 
%%% mode: latex
%%% TeX-master: "rpz"
%%% End: 


\end{document}

%%% Local Variables:
%%% mode: latex
%%% TeX-master: t
%%% End:
